\documentclass[a4paper, draft]{article}
\usepackage[utf8]{inputenc}
\usepackage[french]{babel}
\usepackage{graphicx}
\usepackage{amsmath, amssymb, amsthm}
\usepackage[left=3cm,right=3cm,top=2cm,bottom=2cm]{geometry}
\usepackage{pgf}
\usepackage{tikz}
\usetikzlibrary{arrows,automata,positioning}
\usepackage{verbatim}



\title{Projet : Quoridor}
\author{Montaigu Thomas \\ Rakotosaona Marie-Julie \\ Mattiche Mohamed-said}

\begin{document}

\maketitle

\newpage
\tableofcontents
\newpage

\section{Le Jeu}
Quoridor est un jeu de société où les joueurs ont pour objectif de traverser  le plateau de jeu avec leurs pions. Les joueurs ont la possibilité de placer des murs afin de ralentir la progression des joueurs adverses.

\subsection{Définitions}

\begin{itemize}

\item \textbf{Plateau}: Le plateau est la zone de jeu, il est composé de cases positionnées de façon à former une grille  de taille 9*9 permettant de poser les pions et de zones permettant de poser des murs entre les cases. Le plateau possède deux états disctincts : franchissable et infranchissable. S'il existe au moins un chemin qui permet à chaque pion d'arriver sur sa case finale. 


\item \textbf{Case}: Les cases sont les parties du plateau qui accueuillent les pions, il ne peut y avoir qu'un seul pion sur 1 case.
Les cases sont repérées grâce à une lettre représentant la colonne et un chiffre pour la ligne.


\item \textbf{Mur}: Les murs sont des objets infranchissables par les pions qui peuvent être placés entre les cases. Ils peuvent être positonnés verticalement ou horizontalement. Leurs positions sur le plateau est définie par une case origine  et une direction.
\item \textbf{Case adjacente}: Deux cases sont adjacentes si elles partagent un même côté.
\item \textbf{Case voisine}: Une case est voisine d'une autre si elle est accessible à partir de cette dernière : les deux cases sont adjacentes et ne sont pas séparées par un mur.
\item \textbf{Voisins}: Les voisins d'une case est l'ensemble de ses cases voisines
\item \textbf{Chemin}: Un chemin est une séquence de cases voisines deux à deux. On appellera départ la première case d'un chemin et arrivée sa dernière.
\item \textbf{Chemin valide}: Un chemin est valide s'il permet, à partir de la case de départ, d'arriver sur la ligne de départ de l'adversaire. 
\item \textbf{Case finale}: Une case est finale si pour un pion donné (blanc ou noir), elle appartient à la rangée opposée à l'emplacement de départ du pion 
\end{itemize}

\subsection{Les règles}

Un joueur gagne lorsque son pion atteint un case finale.\\
A chaque tour, un joueur déplace son pion ou place un mur.

Les pions ne se déplacent que d'une case et sur une case adjacente (pas en diagonale) par tour.\\
Un pion ne peut pas acceder à une case bloquée par un mur.

Si sur une case, un pion a pour voisin un autre pion, il peut sauter par dessus la case de cet autre pion. \\
Si cette dernière est déjà prise ou qu'un mur empeche le déplacement, le pion peut se déplacer sur une case adjacente à la case du pion voisin ( en diagonale ).\\
Un mur ne peut etre positionné que si après son positionnement, les deux pions peuvent encore arriver sur la ligne d'arrivée : il existe un chemin valide à partir de chacune d'elles. Le plateau est alors dit franchissable.

\section{Implémentation}
\subsection{Les règles}
Afin de créer un programme pour le jeu, il est nécéssaire de respecter les règles et donc de créer des algorithmes permettant de vérifier ces règles.\\
Les règles à implémenter sont :
\begin{itemize}

  \item Vérifier que les joueurs n'effectuent qu'une seule action par tour. Soit déplacer leur pion, soit placer un mur.
  \item Vérifier que les déplacements des pions sont valides en vérifiant trois points:
      \begin{itemize}
        \item le pion doit rester dans la grille.
        \item le pion ne peut se déplacer que d'une case à une autre, sauf lorsqu'il saute le pions adverse.
        \item vérifier que les pions ne traversent pas de murs lors d'un déplacement.
      \end{itemize}
      \item Vérifier qu'un joueur n'utilise pas plus de murs qu'il n'en possède.
      \item Vérifier que le placement d'un mur n'empeche pas les joueurs d'atteindre la ligne d'arrivée.
      

\end{itemize}

La règle nécessitant l'algorithme le plus complexe est la règle qui impose que lorsqu'un mur est posé il ne doit pas rendre le plateau infranchissable ou bloquer un pion.


\subsection{Représentation des éléménts/structures}

Avant de commencer à réflechir aux algorithmes permettant l'implémentation des règles il nous a fallu réfléchir à la façon dont nous allions définir nos structures
qui représenteraient les différents éléments du jeu. 

Les deux éléments les plus importants et donc ceux à définir de façon la plus complète sont le \textbf{Plateau} et les \textbf{cases}.\\
\textbf{Probleme de representation des ojbets(definition des objets}\\
\begin{itemize}
 
\item \textbf{sous Probleme associes a la representation des murs:}\\
\item comment representer des murs qui sont en accord avec la notation? 
\item Un murs occupe informatiquement la longueur de deux cases,il est construit par deux tranches, comment souder ces deux dernieres?
\item Un mur une fois n est pas valable de le mettre dans une position, il le reste dans toute la partie, comment peut on exploiter ce constat?\\

\item \textbf{sous-Probleme lie au Pion:}\\
\item comme il y a seulement deux pions sur la grille, est-il suffisant d enregistrer leurs places independamment de la structure des case?\\

\item \textbf{ sous-Probleme lie a la grille:}\\
\item la conception concrete et spatiale de la grille separe les cases des pions aux cases des murs voisins, est-il convenable de modeliser la grille informatique ainsi?
\item comment adapter la grille aux notations du probleme?
\item comment optimiser l obtion des donnees les plus frequentes sans le parcours de la grille ?
\item comment faire la liaison entre les murs et pions?\\
\end{itemize}
\textbf{La solution eventuelle de ces sous-Problemes:}\\
\begin{itemize}
\item \textbf{LA CASE:}\\
La case est representee par une structure regroupant les champs suivants:\newline
bas, droite:entier ; ils valent 1 si le mur est present resprectievent en bas, en haut de la case 0 sinon.\newline
basvalid, hautvalid:entiers; ils valent un s il est valide de poser le mur respectivement en bas, en haut de la case.\newline
pion:entier ; il marque si la presence du pion, vaut 1 pour le pion NOIR, 0 pour le BLANC, -1 s il n y a pas de pion.\newline
basorigine, droiteorigine : entiers; si un mur bloque respectivement la case adjacente bas ou celle a droite et et  elle est origine de sa construction vaut 1 sinon 0 \\

\textbf{LA GRILLE:}
\item  La grille est une structure regroupant les champs suivants:\newline
board[C,L]:tableau de deux dimensions de case. \newline
actujoueur:entier;le joueur actuel recupere a chaque tour.\newline
posiPion[2,2], ancPosiPion[2,2]:tableau d entiers; posiPion[0] memorise les coordonnees du pion BLANC, posiPion[1] memorise les coordonnees du pion NOIR et ancPosiPion memorise de la meme facon les coordonnees de la position ancienne.\newline
mur:tableau d entiers; memorise le reste des murs pour les deux joueurs.\newline
\end{itemize}



\begin{verbatim}
            
structure Case { pion : entier 
                 bas, droite : entier
                 basvalid, droite:entier
                 basorigine, originedroite: entier    }                   
                                                                
                                                  
                     
                      

structure board :{   board[9][9] : Tableau de Case
                     posiPion[2][2] : tableau d entier
                     ancPosiPion[2][2]  : entier
                     PositionPionBlanc[2][2] : tableau  d'entier
                     mur[2]: tableau  d'entier  
                     tour:entier    
                     actujoueur:entier   
                   }



\end{verbatim}
\textbf{justification:}\\
\begin{itemize}
\item
Pour les mur, avec les positions bas et droite on peut representer aussi les murs haut et gauche par rapport a une case donnee en se deplacer vers la case voisine gauche et haute et mettre les murs bas et droite par rapport a cette derniere.le choix de ces deux directions bas et haut est deja fixe par le probleme.les champs:basvalid et droitevalid sont indispensables, ils nous permettent de memoriser une fois pour toute si un mur est invalide d etre poser au dessus ou au dessous de la case.Ainsi, cela nous permet d identifier les positions des murs invalides dans un etat donne de la grille, cas d un mur bloquant le chemin par exemple.L identification de l origine du mur, en plus de la convention de construction d un mur(gauche ver droite, haut vers le bas) nous permet d idendifier ses deux tranches.Pour mieux le souder on desactive la validite de droitevalid de la case gauche du mur s il est pose verticalement, sinon on desactive basvalid de case haute du mur s il est pose horizontalement.\\
\item
Pour les pions, il est pratique et moins complexe pour la conception humaine de reserver toute une matrice pour les pions, et garder la position des deux pions dehors.Vu la liaison inseparable entre la presence du pion et la presence des murs, il est plus convenable de confondre la grille des pions et des murs en une seule et garder le nombre de murs restants, le joueur qui change d un tour a un autre et la position des pions a cote de la grille.d'ou notre structure.\\

\end{itemize}        
\subsection{Les algorithmes}



\subsection{Vérifications de la validité des mouvements des pions}

Dans cette section nous allons parler des algorithmes mis en place afin de vérifier la validité des mouvements des pions en controlant les trois points cités précédemment.\\
\vspace{0.25cm}\\
Afin que ces algorithmes aient une compléxité en temps assez faible, nous avons naturellement mis en place une mémoire mémorisant la position du pion blanc et du pion noir
afin de ne pas avoir à re-parcourir la grille de jeu à chaque appel de ces fonctions. Cette mémoire à été rajouté dans la structure définissant le Plateau.\\
\vspace{0.15cm}


Nous avons écrit un algorithme pour chaque point à vérifier et une procédure faisant appel aux trois fonctions afin de décider si oui ou non le mouvement effectué est valide.


Dans toutes les fonctions, \verb?l? et \verb?c? représentent la ligne et la colonne où le joueur \verb?j?  veut déplacer le pion, quand \verb?d? représente la direction dans laquelle
le pion est déplacé (en fonction de sa case d'origine).\\
La fonction \verb?verif_deplacement? s'occupe de vérifier que le joueur ne se déplace que d'une ligne ou colonne lors de son déplacement sauf lorsqu'il enjambe le pion adverse,
\verb?reste_dans_plateau? controle tout simplement que le pion reste dans les limites de la grille du plateau. Quant à la fonction \verb?est_case_voisine?, celle ci s'assure que 
la case d'origine du pion et celle d'arrivée soient bien voisine, c'est à dire qu'il n'existe aucun mur les séparants.
\begin{verbatim}

procedure est_deplacement_valide(j,l,c; entiers, P: Plateau):Booléen

     si j=1
         alors Ligne <- Plateau.lignePionNoir
               Colonne <-Plateau.colonnePionNoir
     sinon si j=0
         alors Ligne <- Plateau.lignePionBlanc
               Colonne <- Plateau.colonnePionBlanc

     si Ligne-l = 1
        alors d<-Haut
     sinon si Ligne-l = -1
        alors d<-Bas
     sinon si Colonne-l = 1
        alors d<-Gauche
     sinon si Colonne -l = -1
        alors d<-Droite


        si reste_dans_plateau(j,l,c,P)
           alors si verif_deplacement(j,l,c,P) et est_case_voisine(j,d,P)
                    retourner vrai
        sinon retourner faux



fonction verif_deplacement (j,l,c entiers, P: Plateau ):Booléen

     si j=1
         alors Ligne <- Plateau.lignePionNoir
               Colonne <-Plateau.colonnePionNoir
     sinon si j=0
         alors Ligne <- Plateau.lignePionBlanc
               Colonne <- Plateau.colonnePionBlanc

     si |Ligne-l|= 1 et |Colonne-c|=0
         alors retouner vrai
     sinon si |Ligne-l|=0 et |Colonne-c|=1
         alors retourner vrai

     sinon si |Ligne-l|= 2 et |Colonne-c|=0 et Plateau[l-1][c].pion= j-1
         alors retouner vrai
     sinon si |Ligne-l|= 0 et |Colonne-c|=2 et Plateau[c-1][l].pion= j-1
         alors retourner vrai

    sinon retourner faux
  

fonction restes_dans_plateau (j,l,c: entiers, P: Plateau):Booléen
      
     si l<9 et c<9 et l>=0 et c>=0
        alors retourner vrai
     sinon retourner faux

fonction est_case_voisine (j: entiers,d; Direction,  P: Plateau):Booléen

     si j=1
         alors Ligne <- Plateau.lignePionNoir
               Colonne <-Plateau.colonnePionNoir
     sinon si j=0
         alors Ligne <- Plateau.lignePionBlanc
               Colonne <- Plateau.colonnePionBlanc 
    
     si d=Gauche
        alors si presenceMur a gauche
                 retourner faux
              sinon retourner vrai
     si d=Haut 
        alors si presenceMur en haut
                 retourner faux
              sinon retourner vrai
     si d=Droite  
        alors si presenceMur a droite
                 retourner faux
              sinon retourner vrai
     si d=Bas
        alors si presenceMur en bas
                 retourner faux
              sinon retourner vrai  

          

\end{verbatim}






\subsection{Vérification du non bloquage du jeu}

Pour la fonction qui doit contrôler que l'ajout d'un mur sur le plateau ne rend pas le plateau infranchissable, nous avons décidé de proposer un algorithme récursif.
Cet algorithme questionne  la case selectionnée pour savoir si c'est une case gagnante pour le joueur, si ce n'est pas le cas elle questionnera récursivement ses voisins. Pour que l'algorithme termine, il faut retenir l'ensemble des cases déja visitées.
Cet algorithme revient en  à chercher de manière récursive un chemin menant jusqu'à une case finale du pion.



\begin{verbatim}

fonction mur_valide(P: Plateau valide, l,c: entiers, p:booléen)-> booléen 
    retourner (chemin_existant_PionNoir(P) & chemin_existant_PionBlanc(P))
        
  
fonction chemin_existant_PionNoir(P: Plateau):booléen
        M<-crée_matrice(9,9)
        retourner  chemin_existant(ligne_pion_noir(P), colonne_pion_noir(P), noir,M)


 fonction chemin_existant(ligne : entier , colonne : entier, couleur, M): booleen
           si deja_visité(M,ligne,colonne)
                 retourner Faux
          sinon
                 memorise(M, ligne,colonne)
                 si estCaseFinale(ligne,colonne, couleur)
                        retourner vrai
                sinon
                         retourner
                               ( chemin_existant(ligne_pion_noir+1, colonne_pion_noir,M)
                              ou chemin_existant(ligne_pion_noir-1, colonne_pion_noir,M)
                              ou chemin_existant(ligne_pion_noir, colonne_pion_noir+1,M)
                              ou chemin_existant(ligne_pion_noir+1, colonne_pion_noir-1,M))
 
\end{verbatim}

Complexité de  \verb?chemin_existant_pion_noir? : On fait à chaque appel récursif 4 appels. La complexité de l'algorithme est donc exponentielle. Cependant, comme on retient les cases où l'on a déja effectué le calcul, le nombre d'appels récursifs est majoré par le nombre de cases : $9*9=81$.

\subsection{L'affichage}
Pour pouvoir afficher la partie, le programme d'affichage doit pouvoir afficher une liste de coups passée en paramètres. Sur chaque ligne est donné le coup du premier joueur et le coup du deuxieme.*
Etant donné qu'il existe une convention de notation de coups, nous devons la respecter.\\
Cette convention est la suivante :

\begin{center}
\begin{verbatim}

e8 e2
e7 e3
e7h e2h

\end{verbatim}
\end{center}

Il faut remarquer que lorsqu'un mur est placé, on ne le déplacera plus. De plus, lorsqu'un joueur décide de placer un mur son pion reste immobile. Il faut donc que le programme d'affichage retienne l'état de la partie en cours.
\\Pour cela nous introduisons la matrice 9*9 d'éléments qui permettra de retenir l'état du jeu. Chaque élément a un champ pour retenir la case,  et un champ pour le mur.
\\ Le programme est basé sur les deux étapes :
\begin{itemize}

  \item Traduction et mise en mémoire d'un ligne de jeu dans la matrice d'éléments.
  \item Affichage de la matrice et du plateau

\end{itemize}

\begin{center}
\begin{verbatim}
procedure affichage(F: fichier_de_coups)
    debut
       etatMatrice<-créeMatrice(9,9,element)
       ligne<-lecture(F)
           tant que (nonVide(ligne))
                traduire(ligne,etatMatrice)
                affiche(etatMatrice)
                ligne<-lecture(F)   
    fin
\end{verbatim}
\end{center}


\section{Les stratégies}
\subsection{aléatoire}

La création de la stratégie aléatoire fut relativement simple étant donnée que celle si ne fait pas appel à des prises de décisions réfléchis, tout est aléatoire. Elle nous a cependant
permis de tester et valider une partie des fonctions réalisées dans ce projet (notamment les fonctions de déplacement de pion (\verb?place_wall?) et de positionnement d'un mur (\verb?place_wall?).
Cette 'stratégies ne faisant pas de

\section{Critiques}
\subsection{Notre Affichage}
Même si l'utilisation de la bibliothèque ncurses nous à permis de créer un affichage plus agréable et moins basique qu'en utilisant des bibliothèques plus basique,
il reste des points améliorables. Premierement il manque des couleurs, nous aurions aussi pu ajouter l'affichage des murs restants de chaque joueurs.
\subsection{Notre serveur}

La plupart des fonctions demandées pour l'implémentation du serveur ont put être réalisées, cependant la fonction la plus complexe n'as pas pu être validée.
En effet même si l'algorithme peu sembler bon, les cas terminaux n'ont pas été suffisemment bien définis et établis, ce qui à fait que lors de l'écriture en code C de 
la fonction nous nous sommes retrouvés face à plusieurs problèmes qui varient et qui persistent encore :  
\begin{itemize}
  \item la vérification ne termine pas.
  \item La fonction effectue parfois des vérifications en dehors de la grille ce qui provoque une erreur de segmentation.
  \item La fonction retourne que le le plateau est toujours valide même dans des cas où il ne l'est pas. 
  \end{itemize}

\subsection{Retour sur le travail réalisé}
Nous avons mis beaucoup de temps à nous mettre d'accord sur la façon dont il fallait représenter les éléments du jeu, nous avons aussi eu du mal à commencer rapidement le projet
à cause de certains désaccords sur la compréhension des consignes et directives données.

Un autre problème fut la répartition des tâches à effectuer, nous avons réussi à séparer quelques tâches, cependant il arrivait parfois que deux personnes aient effectué le même travail.

\end{document}
